\chapter*{}
%\thispagestyle{empty}
%\cleardoublepage

%\thispagestyle{empty}

\begin{titlepage}
 
 
\setlength{\centeroffset}{-0.5\oddsidemargin}
\addtolength{\centeroffset}{0.5\evensidemargin}
\thispagestyle{empty}

\noindent\hspace*{\centeroffset}\begin{minipage}{\textwidth}

\centering
%\includegraphics[width=0.9\textwidth]{imagenes/logo_ugr.jpg}\\[1.4cm]

%\textsc{ \Large PROYECTO FIN DE CARRERA\\[0.2cm]}
%\textsc{ INGENIERÍA EN INFORMÁTICA}\\[1cm]
% Upper part of the page
% 

 \vspace{3.3cm}

%si el proyecto tiene logo poner aquí
%\includegraphics{portada/imagenes/logo.png} 
% \vspace{0.5cm}

% Title

{\Huge\bfseries Implementación de una blockchain resistente a ataques criptográficos cuánticos\\
}
\noindent\rule[-1ex]{\textwidth}{3pt}\\[3.5ex]
{\large\bfseries Subtítulo del proyecto.\\[4cm]}
\end{minipage}

\vspace{2.5cm}
\noindent\hspace*{\centeroffset}\begin{minipage}{\textwidth}
\centering

\textbf{Autor}\\ {María Victoria Granados Pozo}\\[2.5ex]
\textbf{Director}\\
{Gabriel Maciá Fernández\\
Francisco Javier Lobillo Borrero 
}\\[2cm]
%\includegraphics[width=0.15\textwidth]{imagenes/tstc.png}\\[0.1cm]
%\textsc{Departamento de Teoría de la Señal, Telemática y Comunicaciones}\\
%\textsc{---}\\
Granada, 18 de noviembre de 2020
\end{minipage}
%\addtolength{\textwidth}{\centeroffset}
\vspace{\stretch{2}}

 
\end{titlepage}






\cleardoublepage
\thispagestyle{empty}

\begin{center}
{\large\bfseries Implementación de una blockchain resistente a ataques criptográficos cuánticos}\\
\end{center}
\begin{center}
María Victoria Granados Pozo\\
\end{center}

%\vspace{0.7cm}
\noindent{\textbf{Palabras clave}: algoritmo criptográfico, cadena de bloques, computación cuántica, firma, verificación}\\

\vspace{0.7cm}
\noindent{\textbf{Resumen}}\\

Debido al auge de la tecnología en nuestra sociedad cabe pensar cuanto de seguras son las actividades que realizamos con el móvil o un ordenador, por ejemplo, las transacciones bancarias. Por otra parte, los avances tecnológicos nos advierten sobre la seguridad de los algoritmos criptográficos actuales, puesto que estos avances permiten una mayor capacidad de cómputo. De aquí surge la idea de este proyecto, marcando como objetivo evitar que las transacciones almacenadas en una \textit{blockchain} se mantengan a salvo cuando esté disponible la computación cuántica.\\ 

Los algoritmos criptográficos que se utilizan hoy día para la firma y verificación de mensajes, basan su seguridad en la hipótesis de que no se pueden encontrar las claves por fuerza bruta. Así con un computador cuántico la seguridad de dichos algoritmos se vería perjudicada. De esta forma surgen los algoritmos criptográficos post-cuánticos, o dicho de otra manera, algoritmo criptográficos resistentes a un criptoanálisis mediante algoritmos implementables en ordenadores cuánticos.\\

En este proyecto el algoritmo que se va a utilizar es el algoritmo \acrshort{uov} (\textit{Unbalance Oil and Vinegar}). Este algoritmo es resistente a ataques cuánticos puesto que si se considera el problema de la creación y validación de firmas es necesario resolver un sistema con m ecuaciones y n variables, siendo esto un problema NP-duro. De este algoritmo hay que resaltar la simplicidad de las operaciones utilizadas, ya que se firman y verifican los bloques con operaciones de suma y multiplicación de valores pequeños. Para llevar a cabo la implementación del algoritmo UOV ha sido necesario implementar la aritmética del cuerpo finito de 128 elementos.\\

Además de la computación cuántica, la otra tecnología que se ha utilizado son las \textit{blockchain}. Una \textit{blockchain} es una cadena formada por bloques, donde cada bloque almacena información como transacciones o el \textit{hash} del bloque anterior. Este tipo de estructura de almacenamiento permite verificar validar y rastrear todo tipo de información, y cada bloque tiene una única posición en la cadena sin poder ser alterada. Los árboles Merkle se utilizan para realizar una búsqueda eficiente en la \textit{blockchain}. Son árboles donde el \textit{hash} de cada hijo es una combinación de los \textit{hash} de los nodos padre.\\

La única línea de defensa de las \textit{blockchain} es el algoritmo de firma de los bloques, de cara a los computadores cuánticos. Ya que en la actualidad son seguros debido a que los ordenadores clásicos no tienen la capacidad de cómputo necesaria para descifrar cada bloque y obtener la información sin dejar huella.\\

En este proyecto se ha seleccionado la \textit{blockchain} ARK. Debido a sus propiedades de código abierto y su arquitectura modular. Al ser de código abierto cualquier persona puede modificar dicho código y pueden aportar algunas ideas de cambios. Mientras su arquitectura modular permite personalizar la \textit{blockchain} según las necesidades de cada usuario. En nuestro caso cambiar los algoritmos de firma y verificación que vienen por defecto en la \textit{blockchain}, algoritmo ECDSA y Schnorr, por el algoritmo post-cuántico UOV.\\

Para la realización del proyecto se ha usado un contenedor, docker, de manera que las instalaciones de la \textit{blockchain} no se vean reflejadas en la máquina local, aprovechando la propiedad de aislamiento de docker. Por tanto dentro del docker se encontrará el entorno necesario para la instalación de la \textit{blockchain} modificada con los nuevos algoritmos de firma y verificación. Concluyendo en la \textit{blockchain} de ARK  menos vulnerable frente ataques cuánticos, puesto que se ha modificado el algoritmo de firma por el algoritmo UOV. Como valor añadido se podrá integrar en otra \textit{blockchain}, gracias a las propiedades de ARK.\\

\cleardoublepage


\thispagestyle{empty}


\begin{center}
{\large\bfseries Implementation of a blockchain resistant to quantum cryptographic attacks}\\
\end{center}
\begin{center}
María Victoria Granados Pozo\\
\end{center}

%\vspace{0.7cm}
\noindent{\textbf{Keywords}: blockchain, cryptographic algorithm, quantum attacks, quantum computing, sign, verify}\\

\vspace{0.7cm}
\noindent{\textbf{Abstract}}\\

Due to the rise of technology in our society, it is possible to think how safe the activities we carry out with the mobile phone or a computer are, for example, banking transactions. On the other hand, technological advances warn us about the security of current cryptographic algorithms, since these advances allow for greater computing capacity. This is where the idea of this project arises, with the objective of preventing transactions stored in a blockchain from being safe where quantum computer is available.\\

The cryptographic algorithms that are currently used for the signature and verification of messages, base their security on the hypothesis that the keys cannot be found by brute force. With a quantum computer, the security of these algorithms will be broken, mainly due to the calculation capacity of these computers. In this way, post-quantum cryptographic algorithms emerge, i. e., cryptographic algorithms resistant to cryptanalysis using algorithms that can be implemented in quantum computers.\\

In this project the algorithm considered is the \acrshort{uov} (Unbalance Oil and Vinegar) algorithm. This algorithm is resistant to quantum attacks. If we consider the problem of creation and verification of signatures, we will need to solve a system with $m$ equations and $n$ variables, this being an NP-hard problem. In this algorithm, the simplicity of the operations used must be highlighted, since the blocks are signed and verified with operations of addition and multiplication of small values. To carry put the implementation of the UOV algorithm, it has been necessary to implement the finite body arithmetic of 128 element.\\


Besides quantum computing, the other technology that has been used is blockchain. A blockchain is a chain made up of blocks, where each block storage information like transaction or the hash of the previous block. This type of storage structure allows to verify, validate and track all types of information, and each block has just one place. Merkle trees are used to perform an efficient search on the blockchain. They are trees where each child node hash is a combination of the hashes of the parent node.\\

The main line of defense for blockchains is the block signature algorithm. Thus, they are currently safe because classic computers do not have the necessary to computing capacity to decipher each block and obtain the information without leaving a trace.\\

In this project, the ARK blockchain has been selected. Because of its open source properties and its modular architecture. Being open source, anyone can modify the code and can contribute some ideas for changes. Its modular architecture allows to customize the blockchain according to the needs of each user. In this case, we change the signature and verification algorithms that come by default in the blockchain, ECDSA and Schnorr algorithms, for the UOV post-quantum algorithm.\\

To implement the project, a docker container, has been used so that the blockchain installations are not reflected in the local machine, taking advantage of the isolation property of docker. Therefore, within the docker you will find the necessary environment for the installation of the modified blockchain with the new signature and verification algorithms. Concluding in the ARK blockchain is less vulnerable to quantum attacks, since the signature algorithm. As an added value, it can be integrated into another blockchain, thanks to the ARK properties.\\




\chapter*{}
\thispagestyle{empty}

\noindent\rule[-1ex]{\textwidth}{2pt}\\[4.5ex]

Yo, \textbf{María Victoria Granados Pozo}, alumna de la titulación Doble Grado de Ingeniería Informática y Matemáticas de la \textbf{Escuela Técnica Superior
de Ingenierías Informática y de Telecomunicación y Facultad de Ciencias de la Universidad de Granada}, con DNI 77137043, autorizo la
ubicación de la siguiente copia de mi Trabajo Fin de Grado en la biblioteca del centro para que pueda ser
consultada por las personas que lo deseen.

\vspace{6cm}

\noindent Fdo: María Victoria Granados Pozo

\vspace{2cm}

\begin{flushright}
Granada a 18 de noviembre de 2020 .
\end{flushright}


\chapter*{}
\thispagestyle{empty}

\noindent\rule[-1ex]{\textwidth}{2pt}\\[4.5ex]

D. \textbf{Gabriel Maciá Fernández}, Profesor del Área de Ingeniería Telemática del Departamento de Teoría de la Señal, Telemática y Comunicaciones de la Universidad de Granada.

\vspace{0.5cm}

D. \textbf{Francisco Javier Lobillo Borrero}, Profesor del Área de Matemáticas del Departamento Álgebra de la Universidad de Granada.


\vspace{0.5cm}

\textbf{Informan:}

\vspace{0.5cm}

Que el presente trabajo, titulado \textit{\textbf{ Implementación de una blockchain resistente a ataques criptográficos cuánticos}},
ha sido realizado bajo su supervisión por \textbf{María Victoria Granados Pozo}, y autoriza la defensa de dicho trabajo ante el tribunal
que corresponda.

\vspace{0.5cm}

Y para que conste, expide y firma el presente informe en Granada a 18 de noviembre de 2020 .

\vspace{1cm}

\textbf{Los directores:}

\vspace{5cm}

\noindent \textbf{Gabriel Maciá Fernández \ \ \ \ \ Francisco Javier Lobillo Borrero}

\chapter*{Agradecimientos}
\thispagestyle{empty}

       \vspace{1cm}


En especial agradezco a mis tutores Grabiel Maciá y Javier Lobillo, por el apoyo y la paciencia que han tenido conmigo a lo largos de todos estos meses. También a Javier Tallón por orientarme a elegir el tema de este trabajo.\\

A mis padres, Miguel y Esther, que me han soportado y animado en los momentos más difíciles. A mi hermano, Miguel, por darme fuerza en el día a día en estos momentos tan complicados de la pandemia que nos ha tocado vivir. A mi pareja, Alfonso, por animarme a seguir adelante en este largo camino.\\

Por último, agradecer a los profesores y compañeros que me he encontrado a lo largo de estos cinco años de carrera, que tanto me han enseñado y tantos momentos he compartido con ellos.\\
