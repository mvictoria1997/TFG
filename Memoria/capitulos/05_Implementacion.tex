\chapter{Implementación}
\label{sec:implementacion}
%Aquí se deben proporcionar los detalles de cómo se ha llevado a la práctica el diseño propuesto en el capítulo anterior. Deben identificarse claramente herramientas, tecnologías, equipamientos, etc. utilizados o necesarios para el buen funcionamiento de la solución. Se pueden describir los fragmentos de código más importantes, con el fin de clarificar la funcionalidad que proporcionan. En general este capítulo debe facilitar la reutilización de nuestra solución, por lo que debe estar bien documentada. Puede incluir un manual de uso.

Aquí se proporcionan los detalles de las funciones implementadas para la ejecución algoritmo UOV, además de la implementación de la aritmética en el cuerpo de 128 elementos.\\

\section{Funciones del cuerpo de 128 elementos}
Las funciones referentes a esta sección son la suma, el producto y conversiones de elementos del cuerpo de $2^7$ elementos a elementos del cuerpo de 128 elementos.\\

Los elementos del cuerpo se representarán con vectores de siete componentes para poder facilitar la implementación de la suma y del producto. Esto se refleja en las dos tablas denominadas \texttt{exp} y \texttt{log}, que tienen la estructura que se muestra en los códigos \ref{cod:exp} y \ref{cod:log}, respectivamente. Para la implementación se han usado las variables diccionario de \texttt{python}, puesto que tienen fácil acceso a todas las componentes.

\vspace{0.25cm}

\begin{lstlisting}[language=Python,caption=Tabla para calcular la potencia en el cuerpo, label=cod:exp]
exp = {
   0 : [0, 0, 0, 0, 0, 0, 1],
   1 : [0, 0, 0, 0, 0, 1, 0],
   125 : [1, 1, 0, 0, 0, 0, 1],
   126 : [1, 0, 0, 0, 0, 0, 1]
}

\end{lstlisting}

\begin{lstlisting}[language=Python,caption=Tabla para calcular el logaritmo en el cuerpo, label=cod:log]
log = {
   tuple( [0, 0, 0, 0, 0, 0, 1] ) : ' 0 ',
   tuple( [0, 0, 0, 0, 0, 1, 0] ) : ' 1 ',
   tuple( [1, 1, 0, 0, 0, 0, 1] ) : ' 125 ',
   tuple( [1, 0, 0, 0, 0, 0, 1] ) : ' 126 ',
}
\end{lstlisting}

La suma de dos elementos del cuerpo se ha implementando la puerta $xor$, esto es si las $iésimas$-componentes son iguales entonces la suma vale $0$, en caso contrario vale $1$, código \ref{cod:suma-cuerpo}.

\begin{table}[h]
	\label{tab:rel}
	\begin{center}
	\centering
	\resizebox{\linewidth}{!}{
	\begin{tabular}{p{0.14\linewidth} p {0.1\linewidth}p{0.1\linewidth}p{0.65\linewidth}}
		\textbf{Tipo} & \textbf{Nombre} & \textbf{Variable} & \textbf{Descripción} \\
		\toprule
		int vector & n1 & Input & Elemento para sumar del cuerpo de 128 elementos \\[0.5ex]
		int vector & n2 & Input & Elemento para sumar del cuerpo de 128 elementos\\[0.5ex]
		int vector & suma & Output & Elemento  del cuerpo que almacena la suma de n1 y n2\\[0.5ex]
		\bottomrule
	\end{tabular}}
	\end{center}
	\caption{Parámetros de la función \texttt{suma}}
\end{table}

\vspace{0.25cm}
\begin{lstlisting}[language=Python,caption=Suma de dos elementos del cuerpo, label=cod:suma-cuerpo]
def suma(n1=[], n2=[]):

   suma = []
   for i in range(len(n1)):
      if n1[i] == n2[i]:
         suma += [0]
      else:
         suma += [1]
   return suma
\end{lstlisting}

Para el producto de dos elementos del cuerpo se ha diferenciado el caso en el que uno de los vectores sea $0$ en dicho caso el producto vale 0. Si ninguno de los vectores es $0$ entonces se hace uso de las tablas \ref{cod:exp} y \ref{cod:log}, para trabajar con enteros módulo 127, código \ref{cod:producto-cuerpo}.

\begin{table}[h]
	\label{tab:rel}
	\begin{center}
	\centering
	\resizebox{\linewidth}{!}{
	\begin{tabular}{p{0.14\linewidth} p {0.1\linewidth}p{0.1\linewidth}p{0.65\linewidth}}
		\textbf{Tipo} & \textbf{Nombre} & \textbf{Variable} & \textbf{Descripción} \\
		\toprule
		int vector & n1 & Input & Elemento para multiplicar del cuerpo de 128 elementos \\[0.5ex]
		int vector & n2 & Input & Elemento para multiplicar del cuerpo de 128 elementos\\[0.5ex]
		int & suma & Inner & Almacena la suma de los logaritmos de n1 y n2 módulo 127\\[1.5ex]
		int vector & product & Output & Elemento del cuerpo que almacena el producto de n1 y n2\\[0.5ex]
		\bottomrule
	\end{tabular}}
	\end{center}
	\caption{Parámetros de la función \texttt{product}}
\end{table}

\vspace{0.25cm}
\begin{lstlisting}[language=Python,caption=Producto de dos elementos del cuerpo, label=cod:producto-cuerpo]
def product(n1=[], n2=[]):

   product = []
   if (n1 == [0,0,0,0,0,0,0]) or (n2 == [0,0,0,0,0,0,0]):
      product = [0,0,0,0,0,0,0]
   else:
      suma = (int(log.get(tuple(n1))) + int(log.get(tuple(n2))))%127
      product = exp.get(suma)
   return product
\end{lstlisting}


Las siguientes funciones son necesarias para la resolución del sistema de ecuaciones.


Calcula el vector correspondiente vector en el cuerpo de un entero, código \ref{cod:entero-cuerpo}.


\begin{table}[h]
	\label{tab:rel}
	\begin{center}
	\centering
	\resizebox{\linewidth}{!}{
	\begin{tabular}{p{0.14\linewidth} p {0.1\linewidth}p{0.1\linewidth}p{0.65\linewidth}}
		\textbf{Tipo} & \textbf{Nombre} & \textbf{Variable} & \textbf{Descripción} \\
		\toprule
		int & n1 & Input & Entero del que se desea calcular su vector en el cuerpo\\[1.5ex]
		int vector &  & Output & Elemento del cuerpo de 128 elementos\\[0.5ex]
		\bottomrule
	\end{tabular}}
	\end{center}
	\caption{Parámetros de la función \texttt{F128}}
\end{table}
\vspace{0.25cm}

\begin{lstlisting}[language=Python,caption=Convierte un entero en un elemento del cuerpo, label=cod:entero-cuerpo]
def F128(n1):

   if n1 == 127:
      return [0,0,0,0,0,0,0]
   else:
      return exp.get(n1)
\end{lstlisting}


Al estar en un cuerpo todo elementos tiene inverso y tiene sentido hacer la función \texttt{inverso} de un elemento del cuerpo, la implementación que se ha hecho ha sido hacer la multiplicación por los restantes elementos y comprobando que de como resultado el elemento unidad, en este caso $[0,0,0,0,0,0,1]$, código \ref{cod:inverso-cuerpo}.

\begin{table}[h]
	\label{tab:rel}
	\begin{center}
	\centering
	\resizebox{\linewidth}{!}{
	\begin{tabular}{p{0.14\linewidth} p {0.1\linewidth}p{0.1\linewidth}p{0.65\linewidth}}
		\textbf{Tipo} & \textbf{Nombre} & \textbf{Variable} & \textbf{Descripción} \\
		\toprule
		int vector & n & Input & Elemento para calcular su inverso en el cuerpo de 128 elementos\\[1.5ex]
		int vector &  & Output & Inverso de n\\[0.5ex]
		\bottomrule
	\end{tabular}}
	\end{center}
	\caption{Parámetros de la función \texttt{inverse}}
\end{table}

\vspace{0.25cm}
\begin{lstlisting}[language=Python,caption=Inverso de un elemento del cuerpo, label=cod:inverso-cuerpo]
def inverse(n=[]):

   i = 0
   while product(n, F128(i)) != [0,0,0,0,0,0,1]:
      i += 1
   return F128(i)
\end{lstlisting}

La función \texttt{mayor} compara si el elemento $n1$ es mayor que el elemento $n2$, para ello se convierten los vectores a enteros con la tabla \texttt{log} y se compara que entero es mayor, código \ref{cod:mayor-que-cuerpo}.

\begin{table}[h]
	\label{tab:rel}
	\begin{center}
	\centering
	\resizebox{\linewidth}{!}{
	\begin{tabular}{p{0.14\linewidth} p {0.1\linewidth}p{0.1\linewidth}p{0.65\linewidth}}
		\textbf{Tipo} & \textbf{Nombre} & \textbf{Variable} & \textbf{Descripción} \\
		\toprule
		int vector & n1 & Input & Elemento a comparar con n2 en el cuerpo de 128 elementos\\[0.5ex]
		int vector & n2 & Input & Elemento a comparar con n1 en el cuerpo de 128 elementos\\[1.5ex]
		int & log1 & Inner & Almacena el logaritmo de n1\\[0.5ex]
		int & log2 & Inner & Almacena el logaritmo de n2\\[1.5ex]
		boolean &  & Output & \texttt{True} si n1 es mayor\\
		& & & \texttt{False} en otro caso\\[0.5ex]
		\bottomrule
	\end{tabular}}
	\end{center}
	\caption{Parámetros de la función \texttt{mayor}}
\end{table}

\vspace{0.25cm}

\begin{lstlisting}[language=Python,caption=Compara dos elementos del cuerpo, label=cod:mayor-que-cuerpo]
def mayor (n1=[],n2=[]):

   if n1 == [0,0,0,0,0,0,0]:
      return False
   else:
      log1, log2 = log.get(tuple(n1)), log.get(tuple(n2))
      return int(log1) > int(log2)
\end{lstlisting}
 


Convierte una matriz $n1$ del cuerpo $\mathds{F}_2$ en un vector $n$ del cuerpo de 128 elementos. Como los elementos de la matriz son solo 0 y 1 la conversión se reduce a, si en la matriz hay un 0 añade al vector $n$ el vector $[0,0,0,0,0,0,0]$ y en otro caso $[0,0,0,0,0,0,1]$, código \ref{cod:matrizF2-cuerpo}.

\begin{table}[h]
	\label{tab:rel}
	\begin{center}
	\centering
	\resizebox{\linewidth}{!}{
	\begin{tabular}{p{0.14\linewidth} p {0.1\linewidth}p{0.1\linewidth}p{0.65\linewidth}}
		\textbf{Tipo} & \textbf{Nombre} & \textbf{Variable} & \textbf{Descripción} \\
		\toprule
		int vector & n1 & Input & Matriz con elementos en $\mathds{F}_2$\\[1.5ex]
		int & row & Inner & Almacena las filas generadas de la matriz n\\[1.5ex]
		int vector & n & Output & Matriz en el cuerpo de 128 elementos\\[0.5ex]
		\bottomrule
	\end{tabular}}
	\end{center}
	\caption{Parámetros de la función \texttt{matrix\_F2to128}}
\end{table}

\vspace{0.25cm}
\begin{lstlisting}[language=Python,caption=Matriz de $\mathds{F}_2$ a un elemento del cuerpo 128 elementos, label=cod:matrizF2-cuerpo]
def matrix_F2to128(n1=[]):

   n =[]
   for i in range(len(n1)):
      row =[]
      for j in range(len(n1[0])):
         if n1[i][j] == 0:
            row += [[0,0,0,0,0,0,0]]
         else:
            row += [[0,0,0,0,0,0,1]]
      n += [row]
   return n
\end{lstlisting}


Convierte un vector de matrices $n1$ del cuerpo $\mathds{F}_2$ en una matriz $matrix$ del cuerpo de 128 elementos. Como en la función anterior los elementos de la matriz son solo 0 y 1, se aplica el mismo cambio, código \ref{cod:matriz3dF2-cuerpo}.

\begin{table}[h]
	\label{tab:rel}
	\begin{center}
	\centering
	\resizebox{\linewidth}{!}{
	\begin{tabular}{p{0.14\linewidth} p {0.1\linewidth}p{0.1\linewidth}p{0.65\linewidth}}
		\textbf{Tipo} & \textbf{Nombre} & \textbf{Variable} & \textbf{Descripción} \\
		\toprule
		int vector & n1 & Input & Vector de matriz con elementos en $\mathds{F}_2$\\[1.5ex]
		int & n & Inner & Almacena las matrices del vector matrix\\[0.5ex]
		int & row & Inner & Almacena las filas generadas de la matriz n\\[1.5ex]
		int vector & matrix & Output & Vector de matrices con elementos en el cuerpo de 128 elementos\\[0.5ex]
		\bottomrule
	\end{tabular}}
	\end{center}
	\caption{Parámetros de la función \texttt{matrix3d\_F2to128}}
\end{table}

\vspace{0.25cm}

\begin{lstlisting}[language=Python,caption=Vector de matrices de $\mathds{F}_2$ a una matriz del cuerpo de 128 elementos, label=cod:matriz3dF2-cuerpo]
def matrix3d_F2to128(n1=[]):

   matrix =[]
   for i in range(len(n1)):
      n = []
      for j in range(len(n1[0])):
         row = []
         for k in range(len(n1[0][0])):
            if n1[i][j][k] == 0:
               row += [[0,0,0,0,0,0,0]]
               
            else:
               row += [[0,0,0,0,0,0,1]]
         n += [row]
      matrix += [n]
   return matrix
\end{lstlisting}

\section{Funciones con matrices}

En esta sección de explicarán funciones como la suma de matrices, cálculo de la matriz transpuesta, matriz identidad y el producto de matrices tanto en $\mathds{F}_2$ como en el cuerpo de 128 elementos. Además incluye la función que resuelve el sistema de ecuaciones con el método de Gauss-Jordan.\\


Suma de dos matrices $m1$ y $m2$ del cuerpo de 128 elementos, código \ref{cod:suma-matrix}.

\begin{table}[h]
	\label{tab:rel}
	\begin{center}
	\centering
	\resizebox{\linewidth}{!}{
	\begin{tabular}{p{0.14\linewidth} p {0.1\linewidth}p{0.1\linewidth}p{0.65\linewidth}}
		\textbf{Tipo} & \textbf{Nombre} & \textbf{Variable} & \textbf{Descripción} \\
		\toprule
		int vector & m1 & Input & Matriz a sumar en el cuerpo de 128 elementos\\[0.5ex]
		int vector & m2 & Input & Matriz a sumar en el cuerpo de 128 elementos\\[1.5ex]
		int & row & Inner & Almacena las filas de n\\[1.5ex]
		int vector & m\_suma & Output & Suma de las matrices m1 y m2 en el cuerpo de 128 elementos\\[0.5ex]
		\bottomrule
	\end{tabular}}
	\end{center}
	\caption{Parámetros de la función \texttt{matrix\_sum}}
\end{table}

\vspace{0.25cm}

\begin{lstlisting}[language=Python,caption=Suma de dos matrices con elementos en el cuerpo, label=cod:suma-matrix]
def matrix_sum (m1, m2):

   m_suma = []

   if (len(m1) == len(m2)) and (len(m1[0]) == len(m2[0])):
      for i in range(len(m1)):
         row = []
         for j in range(len(m1[0])):
            row += [suma (m1[i][j], m2[i][j])]
         m_suma += [row]
   return m_suma
\end{lstlisting}


Producto de dos matrices con elementos en el cuerpo de 128 elementos, código \ref{cod:prod-matrix}.

\begin{table}[h]
	\label{tab:rel}
	\begin{center}
	\centering
	\resizebox{\linewidth}{!}{
	\begin{tabular}{p{0.14\linewidth} p {0.1\linewidth}p{0.1\linewidth}p{0.65\linewidth}}
		\textbf{Tipo} & \textbf{Nombre} & \textbf{Variable} & \textbf{Descripción} \\
		\toprule
		int vector & n1 & Input & Matriz a multiplicar en el cuerpo de 128 elementos\\[0.5ex]
		int vector & n2 & Input & Matriz a multiplicar en el cuerpo de 128 elementos\\[1.5ex]
		int & p\_suma & Inner & Almacena la suma de los productos de cada elemento de la fila de n1 y de la columna de n2\\[0.5ex]
		int & row & Inner & Almacena las filas de prod\\[0.5ex]
		int vector & prod & Output & Producto de las matrices m1 y m2 en el cuerpo de 128 elementos\\[0.5ex]
		\bottomrule
	\end{tabular}}
	\end{center}
	\caption{Parámetros de la función \texttt{matrix\_product}}
\end{table}

\vspace{0.25cm}

\begin{lstlisting}[language=Python,caption=Producto de matrices con elementos del cuerpo, label=cod:prod-matrix]
def matrix_product (n1=[], n2=[]):

   prod=[]
   if len(n1[0]) == len(n2):
      for i1 in range(len(n1)):
         row = []
         for j in range(len(n2[0])):
            p_suma = [0,0,0,0,0,0,0]
            for i2 in range(len(n2)):
               p_suma = suma (product(n1[i1][i2], n2[i2][j]), p_suma)
            row += [p_suma]
         prod += [row]
   return prod
\end{lstlisting}

Producto de dos matrices con elementos en el cuerpo $\mathds{F}_2$, como los elementos de la matriz son $0$ y $1$ la suma acomulada se hace con la operación lógica \texttt{xor},  código \ref{cod:prodF2-matrix}.

\begin{table}[h]
	\label{tab:rel}
	\begin{center}
	\centering
	\resizebox{\linewidth}{!}{
	\begin{tabular}{p{0.14\linewidth} p {0.1\linewidth}p{0.1\linewidth}p{0.65\linewidth}}
		\textbf{Tipo} & \textbf{Nombre} & \textbf{Variable} & \textbf{Descripción} \\
		\toprule
		int vector & n1 & Input & Matriz a multiplicar en $\mathds{F}_2$\\[0.5ex]
		int vector & n2 & Input & Matriz a multiplicar en $\mathds{F}_2$\\[1.5ex]
		int & p\_suma & Inner & Almacena la suma de los productos de cada elemento de la fila de n1 y de la columna de n2\\[0.5ex]
		int & row & Inner & Almacena las filas de prod\\[1.5ex]
		int vector & prod & Output & Producto de las matrices m1 y m2 en $\mathds{F}_2$\\[0.5ex]
		\bottomrule
	\end{tabular}}
	\end{center}
	\caption{Parámetros de la función \texttt{matrix\_product\_F2}}
\end{table}

\vspace{0.25cm}

\begin{lstlisting}[language=Python,caption=Producto de matrices con elementos en $\mathds{F}_2$, label=cod:prodF2-matrix]
def matrix_product_F2 (n1=[], n2=[]):

   prod=[]
   if len(n1[0]) == len(n2):
      for i1 in range(len(n1)):
         row = []
         for j in range(len(n2[0])):
            p_suma = 0
            for i2 in range(len(n2)):
               p_suma = (n1[i1][i2] * n2[i2][j]) ^ p_suma
            row += [p_suma]
         prod += [row]
   return prod
\end{lstlisting}

Calcula la matriz transpuesta de la matriz dada como parámetro, código \ref{cod:trans-matrix}.

\begin{table}[h]
	\label{tab:rel}
	\begin{center}
	\centering
	\resizebox{\linewidth}{!}{
	\begin{tabular}{p{0.14\linewidth} p {0.1\linewidth}p{0.1\linewidth}p{0.65\linewidth}}
		\textbf{Tipo} & \textbf{Nombre} & \textbf{Variable} & \textbf{Descripción} \\
		\toprule
		int vector & m & Input & Matriz para calcular su transpuesta\\[1.5ex]
		int & row & Inner & Almacena las filas de trans\\[1.5ex]
		int vector & trans & Output & Matriz transpuesta\\[0.5ex]
		\bottomrule
	\end{tabular}}
	\end{center}
	\caption{Parámetros de la función \texttt{matrix\_transpose}}
\end{table}

\vspace{0.25cm}

\begin{lstlisting}[language=Python,caption=Matriz transpuesta, label=cod:trans-matrix]
def matrix_transpose(m):

   trans = []
   for j in range(len(m[0])):
      row = []
      for i in range(len(m)):
         row += [m[i][j]]
      trans += [row]
   return trans
\end{lstlisting}


Calcula la matriz identidad con una determinada dimensión, $dim$, que se pasa como parámetro, código \ref{cod:identi-matrix}.

\begin{table}[h]
	\label{tab:rel}
	\begin{center}
	\centering
	\resizebox{\linewidth}{!}{
	\begin{tabular}{p{0.14\linewidth} p {0.1\linewidth}p{0.1\linewidth}p{0.65\linewidth}}
		\textbf{Tipo} & \textbf{Nombre} & \textbf{Variable} & \textbf{Descripción} \\
		\toprule
		int & dim & Input & Dimensión de la que calcular la matriz identidad en el cuerpo de 128 elementos\\[1.5ex]
		int & row & Inner & Almacena las filas de matrix\\[1.5ex]
		int vector & matrix & Output & Matriz identidad\\[0.5ex]
		\bottomrule
	\end{tabular}}
	\end{center}
	\caption{Parámetros de la función \texttt{matrix\_identity}}
\end{table}

\vspace{0.25cm}

\begin{lstlisting}[language=Python,caption=Matriz identidad del cuerpo, label=cod:identi-matrix]
def matrix_identity(dim):

   matrix = []

   for i in range(dim):
      row = []
      for j in range(dim):
         if i == j:
            row += [[0,0,0,0,0,0,1]]
         else:
            row += [[0,0,0,0,0,0,0]]
      matrix += [row]
   return matrix
\end{lstlisting}


La función \texttt{matrix\_rref} resuelve de un sistema de ecuaciones con el método de Gauss-Jordan, código \ref{cod:Gauss-matrix}. El sistema de ecuaciones es de la forma de la ecuación \ref{eq:Gauss}.

\begin{equation}\label{eq:Gauss}
A\cdot x=b
\end{equation}

\begin{table}[h]
	\label{tab:rel}
	\begin{center}
	\centering
	\resizebox{\linewidth}{!}{
	\begin{tabular}{p{0.14\linewidth} p {0.1\linewidth}p{0.1\linewidth}p{0.65\linewidth}}
		\textbf{Tipo} & \textbf{Nombre} & \textbf{Variable} & \textbf{Descripción} \\
		\toprule
		int vector & A & Input & Matriz de la ecuación\\[0.5ex]
		int vector & b & Input & Vector de la ecuación\\[1.5ex]
		int vector & M & Inner & Matriz aumentada, combinación de A y b\\[1.5ex]
		int vector & x & Output & Solución del sistema\\[0.5ex]
		\bottomrule
	\end{tabular}}
	\end{center}
	\caption{Parámetros de la función \texttt{matrix\_rref}}
\end{table}

\vspace{0.25cm}

\begin{lstlisting}[language=Python,caption=Método de Gauss-Jordan, label=cod:Gauss-matrix]
def matrix_rref(A, b):

   r = 0
   row = []
   n = len(A)

   #Matriz aumentada, añadir b como una columna
   M = A
   for i in range(len(M)):
      M[i] += b[i]

   for k in range(n):
      #Intercambio de filas para que quede arriba la de menor valor
      for i in range(k, n):
         if mayor(M[i][k], M[k][k]):
            row = M[k]
            M[k] = M[i]
            M[i] = row

      #Hacer ceros
      for j in range(k+1, n):
         q = product(M[j][k], inverse(M[k][k]))
         for m in range(k, n+1):
            M[j][m] = suma(M[j][m], product(q, M[k][m]))

   #Calcular la solución x de abajo arriba
   x = [[[0,0,0,0,0,0,0]] for i in range(n)]

   x[n-1] = [product(M[n-1][n], inverse(M[n-1][n-1]))]
   for i in range(n-1, -1, -1):
      z = [0,0,0,0,0,0,0]
      for j in range(i+1, n):
         z = suma(z, product(M[i][j], x[j][0]))
      x[i] = [product(suma(M[i][n], z), inverse(M[i][i]))]

   return x
\end{lstlisting}

\section{Funciones algoritmo UOV}

La función \texttt{clavePrivada} genera la clave privada de un usuario, para ello genera una matriz triangular superior, $\alpha$, en $\mathds{F}_2$ con valores aleatorios, y un vector, $\beta$, en $\mathds{F}_2$. Los parámetros $m$ y $v$ son el número de variables de aceite y vinagre, respectivamente, código \ref{cod:priv-UOV}.

\vspace{-0.23cm}
\begin{table}[h]
	\label{tab:rel}
	\begin{center}
	\centering
	\resizebox{\linewidth}{!}{
	\begin{tabular}{p{0.14\linewidth} p {0.1\linewidth}p{0.1\linewidth}p{0.65\linewidth}}
		\textbf{Tipo} & \textbf{Nombre} & \textbf{Variable} & \textbf{Descripción} \\
		\toprule
		int vector & m & Input & Número de aceites\\[0.5ex]
		int vector & v & Input & Número de vinagres\\[1.5ex]
		int vector & alpha & Output & Vector de matrices triangulares superiores aleatorias en $\mathds{F}_2$, parte de la clave privada \\[0.5ex]
		int vector & beta & Output & Matriz aleatoria en $\mathds{F}_2$, parte de la clave privada\\[0.5ex]
		\bottomrule
	\end{tabular}}
	\end{center}
	\caption{Parámetros de la función \texttt{clavePrivada}}
\end{table}




\begin{lstlisting}[language=Python,caption=Generación clave privada, label=cod:priv-UOV]
def clavePrivada (m, v):

   alpha, beta = [], []
   n = m+v

   for k in range(m):
      #alpha matriz triangular superior
      aux = []
      for i in range (v):
         row = []
         for j in range (n):
            if i > j:
               row += [0]
            else:
               row += [randint(0,1)]
         aux += [row]
      alpha += [aux]

      #beta
      row = []
      for i in range(n):
         if randint(0,1) == 0:
            row += [randint(0,1)]

      beta += [row]
   return alpha, beta
\end{lstlisting}

La función \texttt{clavePublica} genera la clave pública a partir de la clave privada, calculada con la función anterior, los valores $m$ y $v$ el número de variables de aceite y vinagre, además de una matriz de distorsión, código \ref{cod:pub-UOV}.

\begin{table}[h]
	\label{tab:rel}
	\begin{center}
	\centering
	\resizebox{\linewidth}{!}{
	\begin{tabular}{p{0.14\linewidth} p {0.1\linewidth}p{0.1\linewidth}p{0.65\linewidth}}
		\textbf{Tipo} & \textbf{Nombre} & \textbf{Variable} & \textbf{Descripción} \\
		\toprule
		int vector & m & Input & Número aceites\\[0.5ex]
		int vector & v & Input & Número vinagres\\[0.5ex]
		int vector & alpha & Input & Vector de matrices triangulares superiores en $\mathds{F}_2$, parte de la clave privada \\[0.5ex]
		int vector & beta & Input & Matriz en $\mathds{F}_2$, parte de la clave privada\\[0.5ex]
		int vector & T & Input & Matriz de distorsión de la forma \ref{mat:T}\\[1.5ex]
		int vector & alpha\_pub & Output & Vector de matrices en $\mathds{F}_2$, parte de la clave pública\\[0.5ex]
		int vector & beta\_pub & Output & Matriz en $\mathds{F}_2$, parte de la clave pública\\[0.5ex]
		\bottomrule
	\end{tabular}}
	\end{center}
	\caption{Parámetros de la función \texttt{clavePublica}}
\end{table}

\vspace{0.25cm}

\begin{lstlisting}[language=Python,caption=Generación clave pública, label=cod:pub-UOV]
def clavePublica(m, v, alpha, beta, T):

   alpha_pub = []
   beta_pub = []
   T_trans = matrix_transpose(T[0:v])

   for k in range(len(alpha)):
      aux = matrix_product_F2(T_trans , alpha [k])
      alpha_pub += [matrix_product_F2(aux, T)]


      beta_pub += [matrix_product_F2([beta[k]], T)]

   return alpha_pub, beta_pub
\end{lstlisting}

La función \texttt{signature} calcula la firma del hash de un mensaje, $hashed$, con las claves privadas del usuario, $alpha\_F2$ y $beta\_F2$, el número de aceites y vinagres, $m$ y $v$, el tamaño del cuerpo, $r$ y con la matriz de transición, $T$. Se toman los vinagres aleatorios, se resuelve el sistema para calcular los aceites, y con la ecuación \ref{eq:firma} obtenemos la firma del mensaje, código \ref{cod:firma-UOV}.

\begin{table}[h]
	\label{tab:rel}
	\begin{center}
	\centering
	\resizebox{\linewidth}{!}{
	\begin{tabular}{p{0.14\linewidth} p {0.1\linewidth}p{0.1\linewidth}p{0.65\linewidth}}
		\textbf{Tipo} & \textbf{Nombre} & \textbf{Variable} & \textbf{Descripción} \\
		\toprule
		int vector & hashed & Input & Vector \textit{hash}\\[0.5ex]
		int vector & alpha\_F2 & Input & Vector de matrices triangulares superiores en $\mathds{F}_2$, parte de la clave privada\\[0.5ex]
		int vector & beta\_F2 & Input & Matriz en $\mathds{F}_2$, parte de la clave privada \\[0.5ex]
		int  & m & Input & Número aceites\\[0.5ex]
		int  & r & Input & Dimensión de \\[0.5ex] %TODO 
		int  & v & Input & Número vinagres\\[0.5ex]
		int vector & T & Input & Matriz de distorsión de la forma \ref{mat:T}\\[1.5ex]
		int vector & alpha & Inner & Almecena los correspondientes valores de alpha\_F2 en el cuerpo de 128 elementos\\[0.5ex]
		int vector & beta & Inner &  Almecena los correspondientes valores de beta\_F2 en el cuerpo de 128 elementos\\[0.5ex]
		int vector & vinagre & Inner & Vector aleatorio de tamaño v, contiene los vinagres\\[0.5ex]
		int vector & coef & Inner & Almacena los coeficientes del sistema\\[0.5ex]
		int vector & term & Inner & Almacena los términos del sistema\\[0.5ex]
		int vector & oil & Inner & Almacena la solución del sistema, son los aceites del algoritmo\\[1.5ex]
		int vector & firma & Output & Contiene la firma de \textit{hashed}\\[0.5ex]
		\bottomrule
	\end{tabular}}
	\end{center}
	\caption{Parámetros de la función \texttt{signature}}
\end{table}

\vspace{0.25cm}

\begin{lstlisting}[language=Python,caption=Firma del mensaje, label=cod:firma-UOV]
def signature(hashed, alpha_F2, beta_F2, m, r, v, T):

   alpha = matrix3d_F2to128(alpha_F2)
   beta = matrix_F2to128(beta_F2)

   vinagre = []
   for k in range(v):
      aux = randint(0, 127)
      vinagre += [F128(aux)]
   coef = []
   term = []

   n= m + v
   for k in range(m):
      A = matrix_product([vinagre], alpha[k])
      coef += matrix_sum([A[0][v:n]], [beta[k][v:n]])
      v_suma = suma (matrix_product([A[0][0:v]], matrix_transpose([vinagre]))[0][0], matrix_product([beta[k][0:v]], matrix_transpose([vinagre]))[0][0])
      term += [suma(hashed[k], v_suma)]


   oil = matrix_rref(coef, matrix_transpose([term]))

   aux = []
   aux += vinagre + matrix_transpose(oil)[0]
   firma = matrix_product([aux], matrix_transpose(matrix_F2to128(T))) #T = T.inverse()
   return firma[0]
\end{lstlisting}

Por último la función \texttt{verify} comprueba si la firma, $firma$, de un mensaje, $m$, es correcta, para ello utiliza las claves públicas del usuario y el número de variables de aceite, se va comprobando una a una si se cumple la igualdad \ref{eq:veri}, código \ref{cod:verif-UOV}.

\begin{table}[h]
	\label{tab:rel}
	\begin{center}
	\centering
	\resizebox{\linewidth}{!}{
	\begin{tabular}{p{0.14\linewidth} p {0.1\linewidth}p{0.1\linewidth}p{0.65\linewidth}}
		\textbf{Tipo} & \textbf{Nombre} & \textbf{Variable} & \textbf{Descripción} \\
		\toprule
		int vector & hashed & Input & Vector \textit{hash}\\[0.5ex]
		int vector & firma & Input & Vector con la firma del \textit{hash}\\[0.5ex]
		int vector & alpha\_F2 & Input & Vector de matrices triangulares superiores en $\mathds{F}_2$, parte de la clave privada\\[0.5ex]
		int vector & beta\_F2 & Input & Matriz en $\mathds{F}_2$, parte de la clave privada \\[0.5ex]
		int & m & Input & Número aceites\\[1.5ex]
		int vector & alpha & Inner & Almecena los correspondientes valores de alpha\_F2 en el cuerpo de 128 elementos\\[0.5ex]
		int vector & beta & Inner &  Almecena los correspondientes valores de beta\_F2 en el cuerpo de 128 elementos\\[1.5ex]
		boolean & verif & Output & \texttt{True} si la firma es válida\\
		& & & \texttt{False} en otro caso\\[0.5ex]
		\bottomrule
	\end{tabular}}
	\end{center}
	\caption{Parámetros de la función \texttt{verify}}
\end{table}

\vspace{0.25cm}

\begin{lstlisting}[language=Python,caption=Verificación de la firma, label=cod:verif-UOV]
def verify(hashed, firma, alpha_pub_F2, beta_pub_F2, m):

   verif = True
   alpha_pub = matrix3d_F2to128(alpha_pub_F2)
   beta_pub = matrix3d_F2to128(beta_pub_F2)
   for k in range(m):
      aux_alpha = matrix_product(matrix_product ([firma], alpha_pub[k]), matrix_transpose([firma]))
      aux_beta = matrix_product (beta_pub[k], matrix_transpose([firma]))
      verif = verif and (hashed[k] == matrix_sum(aux_alpha,  aux_beta)[0][0])

   return verif
\end{lstlisting}


%\begin{lstlisting}[language=Python,caption=Generación clave privada, label=cod:priv-UOV]

%\end{lstlisting}



