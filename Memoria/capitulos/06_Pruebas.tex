\chapter{Evaluación y pruebas}

En este capítulo se debe proporcionar una medida objetiva de las bondades y beneficios de la solución propuesta, tanto en términos absolutos, como –en la medida de lo posible- comparándola con otras soluciones. Dependiendo del tipo de proyecto, debe incluir los resultados experimentales obtenidos al probar la solución; también puede incluir una tabla o diagrama de los costes reales del desarrollo, para así establecer conclusiones respecto a la planificación y costes estimados a priori. Finalmente, cuando se trata del desarrollo de una aplicación software, se pueden definir baterías de pruebas a realizar, de modo que en este capítulo se especificarán qué pruebas se han realizado, los resultados esperados y los resultados obtenidos. 
