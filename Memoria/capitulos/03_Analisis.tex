\chapter{Análisis del problema}


\section{Especificación de requisitos}

Debe incluir una clara descripción de las funcionalidades que se esperan alcanzar, así como las restricciones o condicionantes que puedan determinar el diseño o solución adoptada. Tras eso deben especificarse claramente los requisitos.

Los requisitos pueden ser funcionales (e.g. La herramienta debe mostrar las medidas de la red en tiempo real), o no funcionales (e.g. se debe garantizar el acceso seguro a la herramienta; el rendimiento debe ser alto; el consumo de memoria debe ser bajo; etc.)


\section{Análisis}

El objetivo de este apartado es mostrar en diferentes subapartados los diferentes subproblemas que han aparecido al realizar el proyecto, describiendo las alternativas que se han considerado, y justificando las decisiones que se han adoptado. A veces, especialmente cuando los conceptos utilizados en este apartado son extensos, es necesario clarificarlos previamente en el Capítulo de \textit{Introducción} (sección \textit{Contenidos teóricos para la comprensión del proyecto}).   