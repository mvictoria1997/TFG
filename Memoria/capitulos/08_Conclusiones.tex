\chapter{Conclusiones}

Este proyecto parte de la idea de hacer más seguras las cadenas de bloques, puesto que por si mismas no lo son. Para ello es necesario tener un algoritmo de firma de los bloques y evitar dichas vulnerabilidades. Además el algoritmo implementado es resistente a ataques cuánticos, de esta forma no solo hará la \textit{blockchain} más segura antes un ordenador clásico sino también ante un ordenador cuántico.\\

El inicio del proyecto fue el estudio de algunos conceptos fundamentales como son la computación cuántica, cadenas de bloques y el algoritmo UOV. A continuación, se implementó el algoritmo criptográfico UOV, junto con la aritmética de cuerpos finitos. La implementación se ha hecho en dos lenguajes para comparar los tiempos de ejecución, en \texttt{sagemath} y \texttt{python}. Este último será el que se utilice en la \textit{blockchain}. Seguidamente, se ha creado el entorno de trabajo para instalar la \textit{blockchain}, utilizando la tecnología docker. También, se ha procedido a la integración del algoritmo en la \textit{blockchain}.\\

Con todos estos pasos se ha conseguido una cadena de bloques resistentes a ataques cuánticos que servirá para almacenar las transacciones generadas con la aplicación Wallet. Todo este proceso de creación y firma tanto de bloques como de transacciones se pueden visualizar con el \textit{explorer} de ARK.\\


Para complementar este proyecto de cara al futuro, se podría trabajar con la base de datos, que incluya la información de los archivos \texttt{json} generados para almacenar la firma, en lugar de tenerlos en archivos independientes.\\

Otra línea de investigación que podría ser interesante, sería integrar esta \textit{blockchain} en otra para hacerla más segura frente a ataques cuánticos gracias a las propiedades de la cadena de bloques de ARK.\\

