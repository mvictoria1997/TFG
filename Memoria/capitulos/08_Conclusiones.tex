\chapter{Conclusiones}

Este projecto parte de la idea de hacer más seguras las cadenas de bloques, puesto que por si mismas no lo son. Para ello es necesario tener un algoritmo de firma de los bloques y evitar dichas vulnerabilidades. Además el algoritmo implementado es resistente a ataques cuánticos así que no solo hará la \textit{blockchain} más segura antes un ordenador clásico sino también ante un ordenador cuántico.\\

El inicio del proyecto fue el estudio de algunos conceptos básicos como son la computación cuántica, cadenas de bloque y el algoritmo UOV. A continuación, se implementó el algoritmo criptográfico UOV, junto con la aritmética de cuerpos finitos. La implementación se ha hecho en dos lenguajes para comparar los tiempos de ejecución, en \texttt{sagemath} y \texttt{python}. Este último será el que se utilice en la \textit{blockchain}. Seguidamente, se ha creado el entorno de trabajo para instalar la \textit{blockchain}, un docker. También, se ha procedido a la integración del algoritmo en la \textit{blockchain}.\\

Con todos estos pasos se ha conseguido 


De cara al futuro, se podría trabajar con la base de datos, que incluyera los datos de los archivos \texttt{json} generados para almacenar la firma, en lugar de tenerlos en archivos independientes.\\

 







\section{Valoración personal}
Se puede incluir una valoración personal del proyecto (opcionalmente)