\chapter{Introducción}


\section{Motivación y contexto del proyecto}
\label{sec:intro:motivacion} %Esto se pone si queremos hacer referencia a esta sección

En esta parte es importante clarificar las siguientes preguntas: 
\begin{enumerate}
\item ¿Cuál es el problema que pretendemos resolver con este proyecto? Debemos introducir un poco el contexto en el que aparece y describir bien en qué consiste dicho problema. 
\item ¿Por qué es importante dicho problema? Hay que tratar de aportar datos y argumentos para indicar que el problema descrito es relevante en el contexto actual. 
\end{enumerate}

Este trabajo surge de la necesidad de disminuir las vulnerabilidades de los algoritmos criptográficos a ataques cuánticos. Así pues, el objetivo de este trabajo es modificar la implementación de la \textit{Blockchain} ARK para hacerla resistente a ataques cuanticos, con ayuda del sistema criptográfico PICNIC.

El desarrollo del trabajo se basa en dos tecnologías la computación cuántica y las cadenas de bloques. En primer lugar vamos a tratar la computación cuántica, se basa en el uso de cubits y hace posible que existan nuevos algoritmos que puedan resolver problemas con una complejidad mayor.

El origen de la computación cuántica surge de la necesidad de descubrir nuevas tecnologías, debido a que la evolución de la tecnología en los últimos años se ha basado en la reducción del tamaño de los transistores, aumentando así la velocidad del proyecto. El problema es que este proceso tiene un tiene un límite.

El estudio de las tecnologías cuánticas se inició en 1980, al principio se imaginaban ordenadores tradicionales que trabajaran con algunos principios de la mecánica cuántica.

En 1998 se consiguió analiza la información que transportaban los cúbits y ejecutar el algoritmo de búsqueda de Grover.

En 2019 IBM presenta el primer ordenador cuántico para uso comercial, que combina tanto la computación cuántica como tradicional para su utilización en invastigaciones y grandes cálculos.



Así mismo la otra tecnología que vamos a utilizar son las cadenas de bloques. Surgieron en la segunda década del siglo XXI, ha sido motor de cambio en el ámbito digital 





\section{Objetivos del proyecto y logros conseguidos}

Debemos poner claramente el objetivo del proyecto, sin muchos rodeos, para que esté claro desde el principio. A veces podemos tener un objetivo general (amplio) y algunos objetivos específicos (más concretos). 

Adicionalmente, debemos aportar información a los evaluadores y lectores de la memoria de cuál es el trabajo que hemos realizado nosotros en este proyecto. Por ello, se debe poner una lista de los ``logros conseguidos''. Un ejemplo: 

\begin{itemize}
	\item Se ha diseñado un sistema que permite...
	\item Se ha modificado un software existente para conseguir...
	\item Se ha desarrollado una interfaz gráfica para el acceso...
\end{itemize}


El objetivo principal es profundizar en estudio de las cadenas de bloques. Además se estudiará el algoritmo para usarlo en la impletación de la Blockchain y hacerla más segura.


\section{Estructura de la memoria}
Se describirán los capítulos que tiene la memoria, indicando qué contenidos habrá en cada uno de ellos, para permitir al lector situarse ante el documento. 


\section{Contenidos teóricos para la comprensión del proyecto}
En esta sección se realizará el desarrollo de los contenidos teóricos que permitan al lector entender el desarrollo del proyecto. Es importante no enrollarse con cosas que no tienen nada que ver con el proyecto. También es importante no copiar texto de otras fuentes si no están citadas.


Conceptos clave:

Contratos inteligentes: Base de "propiedades inteligentes" que permiten definir mediante códigos de la Blockchain, la forma en la que los dispositivos reaccionan ante eventos que tienen lugar en su entorno. Llevan incorporada una máquina virtual que habilita la codificación y ejecución de programas software para determinar las condiciones sobre el intercambio de activos entre agentes.

Firma electrónica: Sirve para controlar la integridad de los datos y asegurar que la información procede de quien dice ser su remitente, garantiza que la información que se almacena o se envía no ha sido modificada. Controla la auditoría del documento. Se requieren criptosistemas aritméticos

Criptosistemas aritméticos: Cada usuario posee una clave pública y otra privada. El usuario cifrará y descifrará el mensaje con su llave privada y pública, además de la llave pública del usuario que descifrará o que haya cifrado el mensaje.


Resumen o Hash: Es el resultado de aplicar una función que transforma un mensaje que longitud variable en uno de longitud fija, denominada función hash. Es el resultado de calcular el resto módulo n con n la longitud fija. Al aplicar la función hash a un fichero, si se modifica algún dato del mismo cambiará su hash y por tanto se sabrá si ha sido manipulado desde que se envió. Así podemos conseguir la integridad del mensaje

Blockchain: Sistemas de almacenamiento de información que se divide en bloque de datos enlazados mediante los hash. A cada bloque se le asocia un hash y contiene el hash del bloque anterior, creando una lista enlazada, la búsqueda de información no es muy óptima si hay un número elevado de bloques. Para ello existen los árboles merkle.
Los blockchain por si mismo no solucionan los problemas de los sistemas de la información y la comunicación. Pero permiten impulsar modificaciones orientadas a crear soluciones más robustas, implicando conocer donde hay que usar las blockchain y cual es la infraestructura.


Árboles Merkle: Árboles binarios con funciones hash, cada nodo tiene con máximo dos hijos, no hay ciclos. El cálculo de los hash de los padres se hace combinando los hash de los hijos. La integridad se obtiene inclueyendo en los bloques el valor del nodo raiz en lugar de añadir el valor de todos los datos protegidos por los bloques, reducimos además la información de la cabecera.




Bibliografia:

https://es.wikipedia.org/wiki/Computaci%C3%B3n_cu%C3%A1ntica
