\chapter{Evaluación y pruebas}

%En este capítulo se debe proporcionar una medida objetiva de las bondades y beneficios de la solución propuesta, tanto en términos absolutos, como –en la medida de lo posible- comparándola con otras soluciones. Dependiendo del tipo de proyecto, debe incluir los resultados experimentales obtenidos al probar la solución; también puede incluir una tabla o diagrama de los costes reales del desarrollo, para así establecer conclusiones respecto a la planificación y costes estimados a priori. Finalmente, cuando se trata del desarrollo de una aplicación software, se pueden definir baterías de pruebas a realizar, de modo que en este capítulo se especificarán qué pruebas se han realizado, los resultados esperados y los resultados obtenidos. 

Se han llevado a cabo pruebas sobre los tiempos de ejecución del algoritmo criptográfico UOV. Se han comparado dichos tiempos en dos lenguajes de programación \texttt{python} y \texttt{SageMath}. Por otra parte las pruebas de la integración han sido realizar una transacción y ver que esa transacción se encuentra en un bloque.\\


\section{Pruebas algoritmo criptográfico}

El tiempo de ejecución del algoritmo se lo lleva en gran parte la generación de la clave pública. En esta función para cada ${alpha_{pub}}_k$, con $k \in \{1,\cdots, v\}$, es necesario multiplicar tres matrices de órdenes $n \times n$, $m \times n$, y $n \times n$, respectivamente. De esta forma la complejidad de la multiplicación de matrices es de $\mathds{O}(n^3)$ por $v$, se obtiene que la complejidad de la generación de $alpha_{pub}$ es de $\mathds{O}(n^4)$.\\

En el caso de $beta_{pub}$, hay que multiplicar $v$ un vector por de tamaño $n$ por un matriz de dimensiones $n \times n$. Así la complejidad del cálculo de $beta_{pub}$ es de $\mathds{O}(n^3)$.\\

Entonces la complejidad de dicha función es de orden $\mathds{O}(n^4)$.\\

La tabla \ref{tab:tiempos} muestra algunos tiempos de ejecución al calcular la clave pública. Se han comparado los tiempos de ejecución en \texttt{python} y en un lenguaje de programación matemático, \texttt{SageMath}. Vemos que los tiempos en \texttt{python} son considerablemente mayores, más aún con valores de $m$ y $v$ grandes, una razón es por que las funciones de \texttt{python} han sido implementadas y no son de una biblioteca como las de \texttt{SageMath} que siempre son más óptimas.\\ 

\begin{table}[H]
	\begin{center}
		\begin{tabular}{p{0.07\linewidth}p{0.07\linewidth}p{0.3\linewidth}p{0.3\linewidth}}
			\textbf{m} & \textbf{v} & \textbf{Tiempo (mm' ss'')} \texttt{python} & \textbf{Tiempo (mm' ss'')} \texttt{SageMath}\\
			\toprule
				$57$ & $197$ & $3'34.65''$ & $0.012''$\\
				$44$ & $151$ & $1'15.61''$ & $0.007''$\\
				$35$ & $191$ & $29.61''$ & $0.009''$\\ 
				$19$ & $65$ & $2.77''$ & $0.006''$\\
				$10$ & $30$ & $0.17''$ & $0.004''$\\
				$3$ & $3$ & $0.0004''$ & $0.003''$\\
			\bottomrule
		\end{tabular}
	\end{center}
	\caption{Tiempos de generación de la clave pública}
	\label{tab:tiempos}
\end{table}

Para la implementación en la \textit{blockchain} se ha optado por los valores de $m$ y $v$, $3$ y $3$, respectivamente. De esta forma los bloques se firmarán rápidamente y las firmas serán más pequeñas. Así para será más didáctico ya que dichas firmas podrán visualizarse mejor y entender que significa.\\

La ventaja al integrar el algoritmo en \textit{blockchain} será que hará la cadena de bloques más segura, sobre todo cuanto mayores sean los valores de $m$ y $v$. Por contra disminuirá el rendimiento, ya que para esos valores de $m$ y $v$ los tiempos serán más elevados.\\

\section{Pruebas de integración}

Para ver que la integración funciona correctamente se han seguido unos pasos.\\ 

En primer lugar se ha instalado e iniciado el \textit{core}, y se han mostrado los logs del mismo para comprobar que no haya ningún error durante la firma de los bloques que se vayan generando.\\

Posteriormente, se ha abierto la aplicación Wallet para realizar una transacción, en los mismos logs del \textit{core} se ha mostrado que la transacción es válida y que se ha almacenado en un bloque. Esta parte no se ha podido cumplimentar, ya que los bloques con una transacción no se firman correctamente.\\

Por último, en el \textit{explorer} de ARK se buscan la transacción y el bloque en el que se ha almacenado, se puede ver la información referente al bloque como quién lo ha firmado, cuándo se ha creado el bloque o el número de transacciones. Si se desea visualizar la firma de un bloque  hay que acceder a la API y buscar el bloque en cuestión, por la ID.
