\chapter{Manual de usuario}

El manual de usuario se va a realizar para una máquina ubuntu.\\

\section{Instalación de Docker}

En primer lugar instalamos la última versión de docker, \cite{instalacion-docker}. 

Desintalar cualquier versión anterior de Docker.\\

\begin{lstlisting}[language=Bash,caption=Instalación Docker. Parte I, label=cod:suma-cuerpo, style=Consola]
sudo apt-get remove docker docker-engine docker.io containerd runc
\end{lstlisting}


Actualizar los paquetes \texttt{apt} para tener acceso a las últimas actualizaciones e instalar los paquetes que permiten al sistema operativo acceder a los repositorios de Docker a través de HTTPS.\\

\begin{lstlisting}[language=Bash,caption=Instalación Docker. Parte II, label=cod:suma-cuerpo, style=Consola]
$ sudo apt-get update
$ sudo apt-get install apt-transport-https ca-certificates curl gnupg-agent software-properties-common
\end{lstlisting}

Añadir la clave GPG oficial de Docker, la clave GPG es una característica de seguridad para asegurar que el software que se va instalar es auténtico.\\
 
\begin{lstlisting}[language=Bash,caption=Instalación Docker. Parte III, label=cod:suma-cuerpo, style=Consola]
$ curl -fsSL https://download.docker.com/linux/ubuntu/gpg | sudo apt-key add -

OK
\end{lstlisting}

Verificar que obtenemos la clave con la siguiente huella, para ello buscamos la huella con los últimos 8 dígitos, de la misma.\\

\texttt{9DC8 5822 9FC7 DD38 854A  E2D8 8D81 803C 0EBF CD88}\\

\begin{lstlisting}[language=Bash,caption=Instalación Docker. Parte IV, label=cod:suma-cuerpo, style=Consola]
$ sudo apt-key fingerprint 0EBFCD88

pub   rsa4096 2017-02-22 [SCEA]
      9DC8 5822 9FC7 DD38 854A  E2D8 8D81 803C 0EBF CD88
uid           [ unknown] Docker Release (CE deb) <docker@docker.com>
sub   rsa4096 2017-02-22 [S]
\end{lstlisting}

Instalar el repositorio de Docker.\\ 

\begin{lstlisting}[language=Bash,caption=Instalación Docker. Parte V, label=cod:suma-cuerpo, style=Consola]
$ sudo add-apt-repository "deb [arch=amd64] https://download.docker.com/linux/ubuntu $(lsb_release -cs) stable"
\end{lstlisting}

Actualizar los repositorios que se acaban de agregar e instalar la última versión de Docker Engine y Docker Containerd.\\


\begin{lstlisting}[language=Bash,caption=Instalación Docker. Parte VI, label=cod:suma-cuerpo, style=Consola]
$ sudo apt-get update
$ sudo apt-get install docker-ce docker-ce-cli containerd.io
\end{lstlisting}

Verificar que se ha instalado correctamente comprobando la versión de Docker.\\

\begin{lstlisting}[language=Bash,caption=Instalación Docker. Parte VII, label=cod:suma-cuerpo, style=Consola]
$ docker --version

Docker version 19.03.13, build 4484c46d9d
\end{lstlisting}

Algunos comandos útiles para el trabajo con Docker se muestran en el código \ref{cod:cm-docker}.

\begin{lstlisting}[language=Bash,caption=Comandos útiles de Docker, label=cod:cm-docker]
#Muestra los contenedores
$ sudo docker ps

#Lista los contenedores con los IDs
$ sudo docker container ls --all

#Lista las imagenes con los IDs
$ sudo docker images ls --all 

#Guarda los cambios del docker
$ sudo docker commit <ID-CONTAINER> <NOMBRE-NUEVO:ETIQUETA>

#Corre un contenedor, abriendo los puertos indicados
$ sudo docker run -it -p <PUERTO:PUERTO> <NOMBRE:ETIQUETA>

#Elimina un contenedor
$ sudo docker rm <ID-CONTAINER>
\end{lstlisting}

\section{Instalación Blockchain ARK}

En Docker, iniciamos una imagen de ubuntu xenial. Además abrimos los puertos que van a ser necesarios posteriormente 4103, para la API pública, 4102, conexión P2P API paray 4200, para el \textit{explorer}.\\

\begin{lstlisting}[language=Bash,caption=Instalación \textit{blockchain}. Parte I, label=cod:suma-cuerpo, style=Consola]
$ sudo docker run -ti -p 4103:4103 -p 4200:4200 ubuntu:xenial
\end{lstlisting}

Una vez estamos dentro de la máquina Docker instalamos sudo, para poder trabajar en modo administrador desde el usuario que vamos a crear.\\

\begin{lstlisting}[language=Bash,caption=Instalación \textit{blockchain}. Parte II, label=cod:suma-cuerpo, style=Consola]
$ apt-get install sudo
$ adduser deployer

Añadiendo el usuario `deployer' ...
Añadiendo el nuevo grupo `deployer' (1001) ...
Añadiendo el nuevo usuario `deployer' (1001) con grupo `deployer' ...
Creando el directorio personal `/home/deployer' ...
Copiando los ficheros desde `/etc/skel' ...
Introduzca la nueva contraseña de UNIX: ********
Vuelva a escribir la nueva contraseña de UNIX: ********
passwd: contraseña actualizada correctamente
Cambiando la información de usuario para deployer
Introduzca el nuevo valor, o presione INTRO para el predeterminado
	Nombre completo []: 
	Número de habitación []: 
	Teléfono del trabajo []: 
	Teléfono de casa []: 
	Otro []: 
¿Es correcta la información? [S/n] S
\end{lstlisting}

Cambiar el modo del usuario deployer incluyéndolo en el grupo sudo para que sea un superusuario y finalmente, entrar al usuario deployer.\\

\begin{lstlisting}[language=Bash,caption=Instalación \textit{blockchain}. Parte III, label=cod:suma-cuerpo, style=Consola]
$ usermod -aG sudo deployer
$ su - deployer
\end{lstlisting}

Actualizamos los paquete e instalamos algunos nuevos como \texttt{git}, \texttt{curl} y \texttt{yarn}.\\

\begin{lstlisting}[language=Bash,caption=Instalación \textit{blockchain}. Parte IV, label=cod:suma-cuerpo, style=Consola]
$ sudo apt-get update
$ sudo apt-get install git curl yarn jq apt-transport-https
\end{lstlisting}

Instalar las dependencias nvm.\\

\begin{lstlisting}[language=Bash,caption=Instalación \textit{blockchain}. Parte V, label=cod:suma-cuerpo, style=Consola]
$ curl -o- https://raw.githubusercontent.com/creationix/nvm/v0.33.8/install.sh | bash
\end{lstlisting}

Para comprobar que se ha instalado correctamente nos salimos del usuario deployer y volvemos a entrar.\\

\begin{lstlisting}[language=Bash,caption=Instalación \textit{blockchain}. Parte VI, label=cod:suma-cuerpo, style=Consola]
$ command -v nvm

nvm
\end{lstlisting}

Para eliminar nvm
\begin{lstlisting}[language=Bash,caption=Instalación \textit{blockchain}. Parte VII, label=cod:suma-cuerpo, style=Consola]
$ nvm use system
$ npm uninstall -g a_module
$ sudo npm  install -g npm
\end{lstlisting}

Instalar pm2.\\

\begin{lstlisting}[language=Bash,caption=Instalación \textit{blockchain}. Parte VIII, label=cod:suma-cuerpo, style=Consola]
$ sudo apt-get install npm
$ sudo npm i -g pm2
$ ln -s /usr/bin/nodejs /usr/bin/node
\end{lstlisting}

Comando interesantes para trabajar con pm2. Nos servirar para observar si la \textit{blockchain} y el \textit{explorer} están levantados.

\begin{lstlisting}[language=Bash,caption=Comandos útiles de pm2, label=cod:suma-cuerpo, style=Consola]
#Lista los demonios de pm2
$ pm2 list

#Se obtienen los estados de los demonios de pm2
$ pm2 status
\end{lstlisting}

Hay que clonar el directorio deployer de \texttt{@ArkEcosystem}.\\

\begin{lstlisting}[language=Bash,caption=Instalación \textit{blockchain}. Parte IX, label=cod:suma-cuerpo, style=Consola]
$ git clone https://github.com/ArkEcosystem/deployer.git
$ chmod 764 deployer/setup.sh
$ ./setup.sh
\end{lstlisting}

Iniciar la base de datos e instalar tanto la \textit{blockchain} como el \textit{explorer}. Una vez que instalemos el \texttt{core} obtendremos en la salida la dirección y la \textit{passphrase} del wallet Genesis la tenemos que guardar en un archivo para más tarde poder hacer las transacciones, aún así podemos encontrar la \textit{passphrase} junto con la dirección en el archivo \path{/home/<USUARIO>/.bridgechain/testenet/<NOMBRE-BRIDGECHAIN>/genesisWallet.json}. Esto puede tardar unos minutos.
\begin{lstlisting}[language=Bash,caption=Instalación \textit{blockchain}. Parte X, label=cod:suma-cuerpo, style=Consola]
$ sudo service postgresql start
$ sudo apt-get update -y 
$ sudo apt-get install -y libjemalloc-dev
$ ./deployer/bridgechain.sh install-core --config deployer/config.sample.conf --autoinstall-deps --non-interactive
\end{lstlisting}

\begin{figure}[h]
	\centering
	\includegraphics[width=15cm,height=9cm]{figuras/Instalacion_bridgechain.png}
	\caption{Salida tras la instalación del core}
	\label{fig:install-bridge}
\end{figure}

\begin{lstlisting}[language=Bash,caption=Instalación \textit{blockchain}. Parte X, label=cod:suma-cuerpo, style=Consola]
$ ./deployer/bridgechain.sh install-explorer --config deployer/config.sample.conf --skip-deps --non-interactive
\end{lstlisting}



Iniciamos la \textit{blockchain} y el \textit{explorer}.\\

\begin{lstlisting}[language=Bash,caption=Instalación \textit{blockchain}. Parte XII, label=cod:suma-cuerpo, style=Consola]
$ ./deployer/bridgechain.sh start-core --network testnet

==> Starting...
Starting victoriabridgechain-relay... done
Starting victoriabridgechain-forger... done
==> Start OK!


$ ./deployer/bridgechain.sh start-explorer --network testnet
\end{lstlisting}

La imagen \ref{fig:install-explorer} muestra la salida después de haber iniciado el \textit{explorer}. Ese mismo cuadro es el que se visualiza tras ejecutar \texttt{pm2 status} pues muestra tanto si la \textit{blockchain} como el \textit{explorer} están iniciados.

\begin{figure}[h]
	\centering
	\includegraphics[width=14.5cm,height=2cm]{figuras/Instalacion_explorer.png}
	\caption{Salida tras iniciar del \textit{explorer}}
	\label{fig:install-explorer}
\end{figure}

Para visualizar las transacciones en el \textit{explorer}, debemos de abrir un navegador y acceder a la url \texttt{http://NODE\_GENESIS\_IP:EXPLORER\_PORT}, donde el \texttt{EXPLORER\_PORT} es $4200$. La imagen \ref{fig:nav-explorer} muestra las primeras transacciones realizadas, entre ellas se encuentra la transacción inicial al monedero génesis, además del registro de los delegados.

\begin{figure}[h]
	\centering
	\includegraphics[width=14.5cm,height=10.5cm]{figuras/Navegacion_explorer.png}
	\caption{Primeras transacciones en el \textit{explorer}}
	\label{fig:nav-explorer}
\end{figure}

\clearpage

\section{Instalación de la aplicación ARK Desktop Wallet}

Antes de descargar la aplicación ARK Desktop Wallet, en el ordenador local, son necesarias algunas instalaciones previas, como algunos archivos de desarrollo de \texttt{libudev}, \texttt{node 12} y \texttt{yarn}, código \ref{cod:install-wallet}.

\begin{lstlisting}[language=Bash,caption=Instalaciones previas a la aplicación ARK Wallet, label=cod:install-wallet, style=Consola]
$ sudo apt-get install libudev-dev libusb-1.0-0-dev

$ npm install -g n
$ sudo n 12

# Para comprobar que se ha instalado node 12
$ n --version

$ npm install -g yarn
\end{lstlisting}

La descarga de la aplicación se realiza desde el repositorio de github de \texttt{@ArkEcosystem/desktop-wallet} \cite{descargas-wallet}, para el proyecto se ha descargado la última versión del archivo \texttt{.deb}.

\begin{lstlisting}[language=Bash,caption=Instalación de la aplicación ARK Wallet, label=cod:install-wallet, style=Consola]
$ sudo dpkg -i ark-desktop-wallet-linux-amd64-<VERSION>.deb

#Para desintalarlo
$ sudo apt-get remove ark-desktop-wallet
\end{lstlisting}


Una vez que se ha instalado la aplicación, la abrimos y nos encontramos con la imagen \ref{fig:wallet-1}. A continuación, vamos siguiendo los pasos que nos salen en la parte de la derecha de la pantalla.

\begin{figure}[H]
	\centering
	\includegraphics[width=13.5cm,height=7.5cm]{figuras/wallet_1.png}
	\caption{Inicio de Wallet}
	\label{fig:wallet-1}
\end{figure}

\newpage
Así se llega al primer paso, crear un perfil, en el que hay que indicar el nombre y la moneda con la que se quiere trabajar, en este caso se dejan los valores por defecto, además existe la opción de cambiar el avatar del usuario, imagen \ref{fig:wallet-2}.

\begin{figure}[H]
	\centering
	\tcbox[colback=white, colframe=codegray]{\includegraphics[width=6cm,height=7cm]{figuras/wallet_2.png}}
	\caption{Detalles del perfil}
	\label{fig:wallet-2}
\end{figure}

En el siguiente paso aparece un menú con las posibles redes, \textit{mainnet} y \textit{devnet}. Sin embargo, si se desea usar la red \textit{testnet} es necesario crearla pulsando nueva red. Los campos a rellenar son el nombre, una breve descripción y la dirección del servidor con el puerto de la API, \texttt{http:/GENESIS\_NODE\_IP:API\_PORT}, imagen \ref{fig:wallet-3}.

\begin{figure}[H]
	\centering
	\tcbox[colback=white, colframe=codegray]{\includegraphics[width=6cm,height=5cm]{figuras/wallet_3.png}}
	\caption{Configuración de la nueva red}
	\label{fig:wallet-3}
\end{figure}

\newpage
Cuando se ha creado la red aparece una pantalla con los detalles de la misma donde debemos de cambiar la dirección por defecto del explorer y poner \texttt{http://GENESIS\_NODE\_IP:EXPLORER\_PORT}. También se pueden cambiar el nombre de los Tokens y el símbolo. Una vez realizados los cambios, guardamos la red, imagen \ref{fig:wallet-4-5}.

\begin{figure}[H]
	\centering
	\begin{minipage}{0.4\textwidth}
		\includegraphics[width=6.5cm,height=8.5cm]{figuras/wallet_4.png}
	\end{minipage}\vfill
	\begin{minipage}{0.38\textwidth}
		\includegraphics[width=6.3cm,height=8.5cm]{figuras/wallet_5.png}	
	\end{minipage}
	\caption{Detalles de la nueva red}
	\label{fig:wallet-4-5}
\end{figure}

\newpage
Se selecciona la red recien creada, \texttt{victoriaBridgechain} y se avanza al siguiente paso, imagen \ref{fig:wallet-6}.

\begin{figure}[H]
	\centering
	\tcbox[colback=white, colframe=codegray]{\includegraphics[width=6cm,height=7cm]{figuras/wallet_6.png}}
	\caption{Selección de la red}
	\label{fig:wallet-6}
\end{figure}

Para acabar con la creación del usuario, se tiene la opción de personalizar los colores de la interfaz de usuario o cambiar al tema de noche, imagen \ref{fig:wallet-7}.

\begin{figure}[H]
	\centering
	\tcbox[colback=white, colframe=codegray]{\includegraphics[width=6cm,height=6cm]{figuras/wallet_7.png}}
	\caption{Personalización del diseño de la interfaz}
	\label{fig:wallet-7}
\end{figure}

\newpage
La primera vez que se entra al usuario es necesario importar el monedero que se ha creado durante la instalación de la \textit{blockchain}, pulsando en la esquina superior derecha en \texttt{Import Wallet}. No es necesario rellenar los dos campos, se puede importar el monedero introduciendo solamente la \textit{passphrase}, imagen \ref{fig:wallet-8}.

\begin{figure}[H]
	\centering
	\tcbox[colback=white, colframe=codegray]{\includegraphics[width=6cm,height=6cm]{figuras/wallet_8.png}}
	\caption{Importar monedero}
	\label{fig:wallet-8}
\end{figure}

Si se desea se puede poner contraseña al monedero, haciéndolo más seguro, en el ejemplo no se van a poner contraseña a ninguno de los monederos, imagen \ref{fig:wallet-9}.

\begin{figure}[H]
	\centering
	\tcbox[colback=white, colframe=codegray]{\includegraphics[width=6cm,height=4cm]{figuras/wallet_9.png}}
	\caption{Encriptación el monedero}
	\label{fig:wallet-9}
\end{figure}

\newpage
Finalmente, hay que ponerle nombre al monedero, para diferenciarlo de otros monederos y saber cual es el que contiene la cantidad inicial, el nombre que se le va a poner es \texttt{MainWallet}, imagen \ref{fig:wallet-10}.

\begin{figure}[H]
	\centering
	\tcbox[colback=white, colframe=codegray]{\includegraphics[width=6cm,height=5cm]{figuras/wallet_10.png}}
	\caption{Confirmación para crear el monedero}
	\label{fig:wallet-10}
\end{figure}

Para que salga la cantidad de dinero inicial es necesario conectar la aplicación con la \textit{blockchain}. Pulsando en el panel lateral izquierdo en \texttt{Network}, se accede a la configuración de la red y pulsando en \texttt{Connect custom peer} obtenemos la imagen \ref{fig:wallet-11-12}. Rellenamos los campos con \texttt{http://GENESIS\_NODE\_IP} y con el \texttt{API\_PORT}. Cuando se conecte hay que refrescar la página para que se actualice el monedero con el dinero.

\begin{figure}[H]
	\centering
	\begin{minipage}{0.4\textwidth}
		\includegraphics[width=7cm,height=6.5cm]{figuras/wallet_11.png}
	\end{minipage}\hfill
	\begin{minipage}{0.4\textwidth}
		\tcbox[colback=white, colframe=codegray]{\includegraphics[width=6cm,height=6cm]{figuras/wallet_12.png}}
	\end{minipage}
	\caption{Configuración de la conexión peer}
	\label{fig:wallet-11-12}
\end{figure}

\newpage
Para realizar transacciones y poder probar el algoritmo, es necesario crear otro monedero y realizar transacciones entre ambos. De esta forma, hay que acceder a la sección \texttt{My wallets} y pulsar \texttt{Create Wallet}. El primer paso es introducir el nombre del monedero, imagen \ref{fig:wallet-13}.
 
\begin{figure}[H]
	\centering
	\tcbox[colback=white, colframe=codegray]{\includegraphics[width=7cm,height=6.5cm]{figuras/wallet_13.png}}
	\caption{Selección de la dirección del nuevo monedero}
	\label{fig:wallet-13}
\end{figure}

La imagen \ref{fig:wallet-14} muestra la \textit{passphrase} del monedero. Esta \textit{passphrase} hay que guardarla, tal y como se hizó con la \textit{passphrase} de monedero importado, pues será necesaria para realizar transacciones posteriormente.

\begin{figure}[H]
	\centering
	\tcbox[colback=white, colframe=codegray]{\includegraphics[width=6.5cm,height=5cm]{figuras/wallet_14.png}}
	\caption{\textit{Passphrase} o clave privada del monedero}
	\label{fig:wallet-14}
\end{figure}

\newpage
La verificación de la \textit{passphrase} se hace introduciendo tres palabras, la número tres, seis y nueve, si se desea se puede realizar la verifiación introduciendo todas las palabras, imagen \ref{fig:wallet-15}.

\begin{figure}[H]
	\centering
	\tcbox[colback=white, colframe=codegray]{\includegraphics[width=7cm,height=5cm]{figuras/wallet_15.png}}
	\caption{Verificación de la \textit{passphrase}}
	\label{fig:wallet-15}
\end{figure}

Si se desea se puede poner contraseña al monedero, en el ejemplo no es necesario, imagen \ref{fig:wallet-16}.

\begin{figure}[H]
	\centering
	\tcbox[colback=white, colframe=codegray]{\includegraphics[width=7cm,height=5cm]{figuras/wallet_16.png}}
	\caption{Encriptación del monedero}
	\label{fig:wallet-16}
\end{figure}

\newpage
Antes de confirmar la creación del monedero, hay que introducir el nombre del mismo, \texttt{Second Wallet}, para hacer referencia a que no tendrá ningún token hasta que no reciba una transferencia, imagen \ref{fig:wallet-17}.

\begin{figure}[H]
	\centering
	\tcbox[colback=white, colframe=codegray]{\includegraphics[width=7cm,height=5cm]{figuras/wallet_17.png}}
	\caption{Confirmación para crear el segundo monedero}
	\label{fig:wallet-17}
\end{figure}




%\begin{lstlisting}[language=Bash,caption=Instalación \textit{blockchain}. Parte III, label=cod:suma-cuerpo, style=Consola]

%\end{lstlisting}












